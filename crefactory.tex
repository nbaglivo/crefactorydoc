\documentclass[a4paper,oneside,10pt]{article}
\usepackage[utf8]{inputenc}
\usepackage{dcolumn}
\usepackage[spanish]{babel}

\begin{document}

\pagenumbering{arabic}

\title{Plan de Desarrollo de Software}
\author{Federico Otar\'an \and Nicol\'as Baglivo}
\date{\today}
\maketitle

\tableofcontents

\section{Introducción}

Este documento describe la implementación de la metodología de trabajo scrum en la empresa Szpinak para la gestión del desarrollo del proyecto turnOn.

Incluye, junto con la descripción de este ciclo de vida iterativo e incremental para el proyecto, los artefactos o documentos con los que se gestionan las tareas de adquisición y suministro: requisitos, monitorización y seguimiento del avance, así como las responsabilidades y compromisos de los participantes en el proyecto.

\subsection{Propósito de este documento}

Facilitar la información de referencia necesaria a las personas implicadas en el desarrollo del sistema turnOn.

\subsection{Alcance}

Personas y procedimientos implicados en el desarrollo del sistema turnOn.

                  
\section{Descripción General de la Metodología}

\subsection{Metodolog\'ia}
Para la ejecución de este proyecto se utilizar\'a una metodogol\'ia \'agil basada en un ciclo de desarrollo iterativo e incremental de tipo Scrum.

\subsection{Fundamentación}

\begin{itemize}

\item Sistema modular. Las características del sistema turnOn permiten desarrollar una base funcional mínima y sobre ella ir incrementando las funcionalidades o modificando el comportamiento o apariencia de las ya implementadas.                   

\item Entregas frecuentes y continuas al cliente de los módulos terminados, de forma que puede disponer de una funcionalidad básica en un tiempo mínimo y a partir de ahí un     incremento y mejora continua del sistema.

\item Para el cliente resulta difícil precisar cuál será la dimensión completa del sistema, y su crecimiento puede prolongarse en el tiempo suspenderse o detenerse.

\end{itemize}

\subsection{Valores de trabajo}
Los valores que deben ser practicados por todos los miembros involucrados en el desarrollo y que hacen posible que la metodología Scrum tenga éxito son:
\begin{itemize}
\item Autonomía del equipo
\item Respeto en el equipo
\item Responsabilidad y auto-disciplina
\item Foco en la tarea
\end{itemize}

\section{Personas y roles del proyecto}


\subsection{Integrantes del equipo}
\begin{tabular}{|l|l|}
\hline
Nombre&Rol\\
\hline
Federico Otar\'an & Arquitecto/Desarrollador/Scrum Master/Product Owner\\
Nicol\'as Baglivo & Arquitecto/Desarrollador/Scrum Master/Product Owner\\
\hline
\end{tabular}

\subsection{Roles del proyecto}

\begin{itemize}
\item Product Owner
\begin{itemize}
\item Escribe historias de usuario, las prioriza, y las coloca en el Product Backlog.
\item Mantenimiento actualizado del Product Backlog en todo momento durante la ejecución del proyecto.
\item Se asegura de que el equipo Scrum trabaje de forma adecuada desde la perspectiva del negocio.
\end{itemize}
\item Scrum Master
\begin{itemize}
\item Planifica y controla el proyecto.
\item Gestiona las prioridades, coordina las interacciones entre equipos y mantiene al equipo del proyecto enfocado en los objetivos.
\item Establece un conjunto de prácticas que aseguran la integridad y calidad de los entregables del proyecto.
\end{itemize}
\item Arquitecto
\begin{itemize}
\item Define y documenta la solución, asegurándose que esté acorde con el sistema deseado y que además sea la correcta para su soporte y evolución.
\item Se asegura que todos los involucrados estén utilizando la solución elaborada y la estén utilizando bien.
\item Conoce cuales cualidades sistémicas deben alcanzarse y en qué medida.
\item Responde sobre las inquietudes relacionadas con la selección de herramientas y ambientes de desarrollo.
\item Resuelve conflictos y ayuda a generar acuerdos.
\item Mantiene la moral, tanto en el interior del grupo de arquitectura como al exterior.
\item Gerencia las estrategias de identificación y mitigación de los riesgos asociados con la arquitectura.
\end{itemize}
\item Desarrollador
\begin{itemize}
\item Contribuye a la visión general del proyecto a nivel de aplicación.
\item Realiza tareas de programación individuales.
\end{itemize}
\end{itemize}

\section{Artefactos}

\subsection{Product Backlog}
Es el equivalente a los requisitos del sistema en esta metodología.\\
El Product Owner es responsable de su correcta gestión, durante todo el proyecto.
El Product Owner puede recabar las consultas y asesoramiento que pueda necesitar para su redacción y gestión durante el proyecto al Scrum Master del mismo.

\subsection{Sprint Backlog}
Es el documento de registro de los requisitos detallados o tareas que va a desarrollar el equipo técnico en la iteración (actual o que está preparándose para comenzar).

\subsection{Incremento}

Parte o subsistema que se produce en un sprint y se entrega al Product Owner completamente terminada y operativa.

\end{document}




















