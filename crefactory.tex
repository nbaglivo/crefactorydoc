\documentclass[a4paper,oneside,10pt]{article}
\usepackage[utf8]{inputenc}
\usepackage{dcolumn}
\usepackage[spanish]{babel}
\usepackage{graphicx}

\begin{document}

\pagenumbering{arabic}

\title{CRefactory}
\author{Baglivo Nicol\'as C\'esar \and Costa Federico Daniel \and Perera Nicanor Gonzalo \and Szeinfeld Matias Ezequiel \and Tarragona Juan Pablo}
\date{\today}
\maketitle

\tableofcontents

\section{Organizaci\'on del documento}
Contar como esta organizado el documento

\section{Introducción}
El refactoring de c\'odigo es una técnica para la reestructuración del código existente de un producto de software, alterando su estructura interna sin cambiar su comportamiento externo con el objetivo de mejorar atributos no funcionales del software como pueden ser la legibilidad, la mantenibilidad o incluso la extensibidad del mismo. Existen una serie de t\'ecnicas bien conocidas de uso com\'un durante el proceso de refactoring de c\'odigo, CRefactory es un entorno de desarrollo que provee soporte para realizar de manera autom\'atica los aspectos mec\'anicos de dicho proceso mediante una interfaz gr\'afica.

Crefactory esta programado en Smalltalk dentro del entorno de desarrollo VisualWorks. La interfaz gr\'afica que posee la aplicaci\'on permite interactuar con c\'odigo C para facilitar dicha refactorizaci\'on. Provee la carga de archivos .c, y directorios con header files. as\'i como tambien configuraciones a descartar (macros y false conditions).  
Permite realizar refactoring de variables ( cambio de nombre, crear estructuras de las variables), de las estructuras (renombrar campos) y de funciones (renombrar las mismas).


\subsection{Motivaci\'on}
¿ Cu\'al es el problema en general ? ¿ Qu\'e motivo al proyecto ?

CRefactory funcionaba correctamente, pero estaba lejos de ser la herramienta ideal de trabajo, ya que su interfaz no era amigable con el usuario. En primer lugar, era muy poco pr\'actica la carga de archivos, ya que era necesario escribir el path de los archivos a mano, y no se poseia ningun tipo de feedback para saber si se habia referenciado correctamente dicho archivo. Se vi\'o la necesidad de mejorar la manera en que se cargaran los archivos.

En segundo lugar, la herramienta no era capaz de agregar false conditions o macros por medio de la interfaz, con lo cual se desaprovechaba  parte del poder de CRefactory. Era necesario de implementar esto de una manera amigable al usuario por medio de una interfaz gr\'afica.

En tercer lugar, no se entend\'ia que ocurr\'ia en caso de que se produzca una excepci\'on. Se necesitaba implementar una manera f\'acil y amigable para que se traten las mismas y as\'i detectar el error. Se necesitaba un mejor manejo de las excepciones y una interfaz para tratarlas.
Por ejemplo, si ocurr\'ia un error durante el parseo de un archivo, no mostraba el contenido del mismo y por ende no mostraba la l\'nea de c\'odigo que produjo el error. As\'i que tambi\'en se necesito implementar highlighting para la detecci\'on de error en parseo.


\subsection{Objetivos}
Objetivos generales

\section{Reuni\'on uno: Carga de archivos}

\subsection{Contexto}
¿ C\'omo estaba la aplicaci\'on ? ¿ Por qu\'e eso es un problema ?

\subsection{Soluci\'on planteada}
Cu\'al fue la soluci\'on al problema

\section{Reuni\'on dos: Nueva interfaz para la configuraci\'on}

\subsection{Contexto}
¿ C\'omo estaba la aplicacion ? ¿ Por qu\'e eso es un problema ?

\subsection{Soluci\'on planteada}
Cu\'al fue la soluci\'on al problema

\section{Reuni\'on tres: Manejo de excepciones}

\subsection{Contexto}
¿ C\'omo estaba la aplicacion ? ¿ Por qu\'e eso es un problema ?

\subsection{Soluci\'on planteada}
Cu\'al fue la soluci\'on al problema

\section{Reuni\'on cuatro: Ventana de error}

\subsection{Contexto}
¿ C\'omo estaba la aplicaci\'on ? ¿ Por qu\'e eso es un problema ?

\subsection{Soluci\'on planteada}
Cu\'al fue la solucion al problema

\section{Reuni\'on cinco: Highlight del error}

\subsection{Contexto}
¿ C\'omo estaba la aplicaci\'on ? ¿ Por qu\'e eso es un problema ?

\subsection{Soluci\'on planteada}
Cu\'al fue la solucion al problema

\section{Minutas}

\subsection{Reunión del lunes 10/09}
Se vio una introducción al proyecto y un vistazo general de como está organizado.
Se vio como levantar el ambiente.

Se habló sobre las siguientes características que tendríamos que ir implementando a lo largo de la cursada:
\begin{itemize}
  \item Mejorar el diálogo de carga de un archivo. Que sea por ejemplo como el de carga de parcels.
  \item Poder especificar donde estan los directorios con los headers.
  \item Tener en cuenta las configuraciones que no se pueden parsear.
  \item Manejar las excepciones para que cuando hay una conf. inválida muestre en el código donde esta lo inválido.
  \item Pensar en la posibilidad de hacer grafos de la inclusión de archivos.
\end{itemize}

\subsection{Reunión del lunes 17/09}
Quedamos en modificar el diálogo de carga de archivo para que permita explorar el file system. Debería ser similar al de carga de parcels.

\subsection{Reunión del Jueves 18/10}
Se hablaron los siguientes puntos:
\begin{itemize}
 \item Mejorar include Directory => Add Directamente
 \item Marcar si un directorio es ReadOnly.
 \item Excepción si no se incluye los inludeDirectories necesarios. 
 \item Manejo de todas las excepciones posibles.
 \item Bajar un ejemplo opensource en C para probar
 \item Prestar atención a preprocess y a FullNameOfFile
\end{itemize}

\subsection{Reunión del lunes 29/10}
Leer documentación y mirar ejemplos en el código fuente de Visual Works para ver como hacer un correcto manejo de excepciones.
Cuando se produce un error poder mostrar el archivo que lo produjo y hacer highlight en la línea de código.

\subsection{Reunión del 12/11}
Que pasen los tests de CRefactory
Si no puedo parsear una configuración, debo ponerla como condición falsa.  El problema es que el error puede ocurrir en muchos lugares.  Debemos comenzar probando con el test19.c (el que tiene el \#ifdef\_cplusplus).

\subsection{Reunión del 29/11}
Cachear la excepción al agregar False Conditions. Mirar en el preprocessor, en la linea: completeConditionalsFrom: OutputStream


\section{Logros alcanzados}

\section{Conclusi\'on}

\subsection{Valores obtenidos}
\subsection{Trabajo futuro}
En orden de conseguir que CRefactory sea una herramienta profesional que pueda competir con otros productos similares creemos que se debe soportar las siguientes caracteristicas:
\begin{itemize}
	\item Resaltado de sintaxis.
	\item Localizaci\'on de c\'odigo potencialmente refactorizable de manera autom\'atica.
	\item Integraci\'on con frameworks de testing para el c\'odigo.
	\item Personalizaci\'on del entorno.
	\item Formateo autom\'atico de c\'odigo.
\end{itemize}


\end{document}